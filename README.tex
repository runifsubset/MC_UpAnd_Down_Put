% Options for packages loaded elsewhere
\PassOptionsToPackage{unicode}{hyperref}
\PassOptionsToPackage{hyphens}{url}
%
\documentclass[
]{article}
\usepackage{amsmath,amssymb}
\usepackage{iftex}
\ifPDFTeX
  \usepackage[T1]{fontenc}
  \usepackage[utf8]{inputenc}
  \usepackage{textcomp} % provide euro and other symbols
\else % if luatex or xetex
  \usepackage{unicode-math} % this also loads fontspec
  \defaultfontfeatures{Scale=MatchLowercase}
  \defaultfontfeatures[\rmfamily]{Ligatures=TeX,Scale=1}
\fi
\usepackage{lmodern}
\ifPDFTeX\else
  % xetex/luatex font selection
\fi
% Use upquote if available, for straight quotes in verbatim environments
\IfFileExists{upquote.sty}{\usepackage{upquote}}{}
\IfFileExists{microtype.sty}{% use microtype if available
  \usepackage[]{microtype}
  \UseMicrotypeSet[protrusion]{basicmath} % disable protrusion for tt fonts
}{}
\makeatletter
\@ifundefined{KOMAClassName}{% if non-KOMA class
  \IfFileExists{parskip.sty}{%
    \usepackage{parskip}
  }{% else
    \setlength{\parindent}{0pt}
    \setlength{\parskip}{6pt plus 2pt minus 1pt}}
}{% if KOMA class
  \KOMAoptions{parskip=half}}
\makeatother
\usepackage{xcolor}
\usepackage[margin=1in]{geometry}
\usepackage{color}
\usepackage{fancyvrb}
\newcommand{\VerbBar}{|}
\newcommand{\VERB}{\Verb[commandchars=\\\{\}]}
\DefineVerbatimEnvironment{Highlighting}{Verbatim}{commandchars=\\\{\}}
% Add ',fontsize=\small' for more characters per line
\usepackage{framed}
\definecolor{shadecolor}{RGB}{248,248,248}
\newenvironment{Shaded}{\begin{snugshade}}{\end{snugshade}}
\newcommand{\AlertTok}[1]{\textcolor[rgb]{0.94,0.16,0.16}{#1}}
\newcommand{\AnnotationTok}[1]{\textcolor[rgb]{0.56,0.35,0.01}{\textbf{\textit{#1}}}}
\newcommand{\AttributeTok}[1]{\textcolor[rgb]{0.13,0.29,0.53}{#1}}
\newcommand{\BaseNTok}[1]{\textcolor[rgb]{0.00,0.00,0.81}{#1}}
\newcommand{\BuiltInTok}[1]{#1}
\newcommand{\CharTok}[1]{\textcolor[rgb]{0.31,0.60,0.02}{#1}}
\newcommand{\CommentTok}[1]{\textcolor[rgb]{0.56,0.35,0.01}{\textit{#1}}}
\newcommand{\CommentVarTok}[1]{\textcolor[rgb]{0.56,0.35,0.01}{\textbf{\textit{#1}}}}
\newcommand{\ConstantTok}[1]{\textcolor[rgb]{0.56,0.35,0.01}{#1}}
\newcommand{\ControlFlowTok}[1]{\textcolor[rgb]{0.13,0.29,0.53}{\textbf{#1}}}
\newcommand{\DataTypeTok}[1]{\textcolor[rgb]{0.13,0.29,0.53}{#1}}
\newcommand{\DecValTok}[1]{\textcolor[rgb]{0.00,0.00,0.81}{#1}}
\newcommand{\DocumentationTok}[1]{\textcolor[rgb]{0.56,0.35,0.01}{\textbf{\textit{#1}}}}
\newcommand{\ErrorTok}[1]{\textcolor[rgb]{0.64,0.00,0.00}{\textbf{#1}}}
\newcommand{\ExtensionTok}[1]{#1}
\newcommand{\FloatTok}[1]{\textcolor[rgb]{0.00,0.00,0.81}{#1}}
\newcommand{\FunctionTok}[1]{\textcolor[rgb]{0.13,0.29,0.53}{\textbf{#1}}}
\newcommand{\ImportTok}[1]{#1}
\newcommand{\InformationTok}[1]{\textcolor[rgb]{0.56,0.35,0.01}{\textbf{\textit{#1}}}}
\newcommand{\KeywordTok}[1]{\textcolor[rgb]{0.13,0.29,0.53}{\textbf{#1}}}
\newcommand{\NormalTok}[1]{#1}
\newcommand{\OperatorTok}[1]{\textcolor[rgb]{0.81,0.36,0.00}{\textbf{#1}}}
\newcommand{\OtherTok}[1]{\textcolor[rgb]{0.56,0.35,0.01}{#1}}
\newcommand{\PreprocessorTok}[1]{\textcolor[rgb]{0.56,0.35,0.01}{\textit{#1}}}
\newcommand{\RegionMarkerTok}[1]{#1}
\newcommand{\SpecialCharTok}[1]{\textcolor[rgb]{0.81,0.36,0.00}{\textbf{#1}}}
\newcommand{\SpecialStringTok}[1]{\textcolor[rgb]{0.31,0.60,0.02}{#1}}
\newcommand{\StringTok}[1]{\textcolor[rgb]{0.31,0.60,0.02}{#1}}
\newcommand{\VariableTok}[1]{\textcolor[rgb]{0.00,0.00,0.00}{#1}}
\newcommand{\VerbatimStringTok}[1]{\textcolor[rgb]{0.31,0.60,0.02}{#1}}
\newcommand{\WarningTok}[1]{\textcolor[rgb]{0.56,0.35,0.01}{\textbf{\textit{#1}}}}
\usepackage{graphicx}
\makeatletter
\def\maxwidth{\ifdim\Gin@nat@width>\linewidth\linewidth\else\Gin@nat@width\fi}
\def\maxheight{\ifdim\Gin@nat@height>\textheight\textheight\else\Gin@nat@height\fi}
\makeatother
% Scale images if necessary, so that they will not overflow the page
% margins by default, and it is still possible to overwrite the defaults
% using explicit options in \includegraphics[width, height, ...]{}
\setkeys{Gin}{width=\maxwidth,height=\maxheight,keepaspectratio}
% Set default figure placement to htbp
\makeatletter
\def\fps@figure{htbp}
\makeatother
\setlength{\emergencystretch}{3em} % prevent overfull lines
\providecommand{\tightlist}{%
  \setlength{\itemsep}{0pt}\setlength{\parskip}{0pt}}
\setcounter{secnumdepth}{-\maxdimen} % remove section numbering
\ifLuaTeX
  \usepackage{selnolig}  % disable illegal ligatures
\fi
\IfFileExists{bookmark.sty}{\usepackage{bookmark}}{\usepackage{hyperref}}
\IfFileExists{xurl.sty}{\usepackage{xurl}}{} % add URL line breaks if available
\urlstyle{same}
\hypersetup{
  pdftitle={Tarification par Monte Carlo d'une Option Put européenne de type Up-and-Out},
  pdfauthor={Marcel Héritier K.},
  hidelinks,
  pdfcreator={LaTeX via pandoc}}

\title{Tarification par Monte Carlo d'une Option Put européenne de type
Up-and-Out}
\author{Marcel Héritier K.}
\date{2023-12-21}

\begin{document}
\maketitle

\hypertarget{uxe0-propos}{%
\subsection{à propos}\label{uxe0-propos}}

Dans ce tutoriel, nous voyons comment utiliser la méthode de simulation
de Monte Carlo pour évaluer les produits financiers dérivés. La notation
mathématique et les exemples sont tirés du livre Implementing
Derivatives Models de Les Clewlow et Chris Strickland.

La valorisation des produits financiers dérivés par des simulations de
Monte Carlo n'est possible qu'en utilisant les mathématiques financières
de l'évaluation neutre du risque et en simulant des trajectoires
d'actifs neutres du point de vue du risque. La formule de prix de
l'espérance risque-neutre en temps continu est donnée par :

\protect\hyperlink{_}{\includegraphics{https://latex.codecogs.com/svg.latex?\%5C\%5C\%20\%5Cbegin\%7Bequation\%7D\%5CLARGE\%20\%5C\%5C\%20\%5Cfrac\%7BC_t\%7D\%7BB_t\%7D\%20\%3D\%20\%5Cmathbb\%7BE\%7D_\%7B\%5Cmathbb\%7BQ\%7D\%7D\%5B\%5Cfrac\%7BC_T\%7D\%7BB_T\%7D\%5Cmid\%20F_t\%5D\%20\%5C\%5C\%20\%5Cend\%7Bequation\%7D}}

\hypertarget{dynamique-dun-mbg}{%
\section{Dynamique d'un MBG}\label{dynamique-dun-mbg}}

Ici, nous disposons d'un taux d'intérêt constant, le facteur
d'actualisation est donc \(\exp(-rt)\), et la dynamique des actions est
modélisée par le mouvement brownien géométrique (GBM).

La solution suivante s'applique au prix de l'action S dans le cadre
d'une dynamique neutre à l'égard du risque : \[
S_{t+\Delta t}=S_{t}\exp\left(\nu\Delta t+\sigma\sqrt{\Delta t}\epsilon_{i}\right)
\] où \(\nu=r-\frac{1}{2}\sigma^{2}\)

\hypertarget{particularituxe9s-des-options-barriuxe8res}{%
\subsection{Particularités des options
barrières}\label{particularituxe9s-des-options-barriuxe8res}}

Lors de la détermination du prix d'options complexes ou exotiques
dépendant d'un scénario, un produit très répandu est l'option à
barrière. Il s'agit d'options européennes standard à expiration, mais
elles cessent d'exister ou n'existent que si le prix du sous-jacent
franchit une barrière prédéterminée. Ce niveau de barrière peut avoir un
seuil de déclenchement continu ou discret \(\tau\).

Pour une option \emph{barrière} du type \emph{Up-and-out}, nous avons :

\[
C_{T}=f(S_{T})=(K-S_{T})^{+}\times\mathbb{I}_{\left\{ \max\limits _{t\in\tau}S_{t}<H\right\} }
\] Since, for \(m\in M\),

\begin{itemize}
\tightlist
\item
  Si \(t\in\tau\) et \(S_t\geq H\), alors \(C_T = 0\)
\item
  Sinon \(t\notin\tau\) et \(S_t < H\), alors
  \(C_T = \max (0,\quad K-S_T)\)
\end{itemize}

\begin{Shaded}
\begin{Highlighting}[]
\NormalTok{S0 }\OtherTok{=} \DecValTok{100}      \CommentTok{\# initial stock price}
\NormalTok{K }\OtherTok{=} \DecValTok{100}       \CommentTok{\# strike price}
\NormalTok{T }\OtherTok{=} \DecValTok{1}         \CommentTok{\# time to maturity in years}
\NormalTok{H }\OtherTok{=} \DecValTok{125}       \CommentTok{\# up{-}and{-}out barrier price/value}
\NormalTok{r }\OtherTok{=} \FloatTok{0.01}      \CommentTok{\# annual risk{-}free rate}
\NormalTok{vol }\OtherTok{=} \FloatTok{0.2}     \CommentTok{\# volatility (\%)}

\NormalTok{N }\OtherTok{=} \DecValTok{100}       \CommentTok{\# number of time steps}
\NormalTok{M }\OtherTok{=} \DecValTok{1000}      \CommentTok{\# number of simulations}
\end{Highlighting}
\end{Shaded}

\hypertarget{simulation}{%
\subsection{Simulation}\label{simulation}}

Ici, nous simulons le prix de l'action \(S_t\) directement, car nous
avons besoin de cette valeur lors du calcul pour la comparer à la
barrière \(H\).

\begin{Shaded}
\begin{Highlighting}[]
\CommentTok{\#{-}{-}{-}{-}{-}{-}{-}{-}{-}{-}{-}{-}{-}{-}{-}{-}{-}{-}{-}{-}{-}{-}{-}{-}\#}
\CommentTok{\# Slow Solution {-} Steps  \#}
\CommentTok{\#{-}{-}{-}{-}{-}{-}{-}{-}{-}{-}{-}{-}{-}{-}{-}{-}{-}{-}{-}{-}{-}{-}{-}{-}\#}
\CommentTok{\# Initialize variables}

\NormalTok{dt }\OtherTok{\textless{}{-}}\NormalTok{ T}\SpecialCharTok{/}\NormalTok{N}
\NormalTok{nudt }\OtherTok{\textless{}{-}}\NormalTok{ (r }\SpecialCharTok{{-}} \FloatTok{0.5} \SpecialCharTok{*}\NormalTok{ vol}\SpecialCharTok{\^{}}\DecValTok{2}\NormalTok{) }\SpecialCharTok{*}\NormalTok{ dt}
\NormalTok{volsdt }\OtherTok{\textless{}{-}}\NormalTok{ vol }\SpecialCharTok{*} \FunctionTok{sqrt}\NormalTok{(dt)}
\NormalTok{erdt }\OtherTok{\textless{}{-}} \FunctionTok{exp}\NormalTok{(r }\SpecialCharTok{*}\NormalTok{ dt)}

\CommentTok{\# Standard Error Placeholders}
\NormalTok{sum\_CT }\OtherTok{\textless{}{-}} \DecValTok{0}
\NormalTok{sum\_CT2 }\OtherTok{\textless{}{-}} \DecValTok{0}

\CommentTok{\# Monte Carlo Method}
\ControlFlowTok{for}\NormalTok{ (i }\ControlFlowTok{in} \DecValTok{1}\SpecialCharTok{:}\NormalTok{M) \{}
  \CommentTok{\# Barrier Crossed Flag}
\NormalTok{  BARRIER }\OtherTok{\textless{}{-}} \ConstantTok{FALSE}
\NormalTok{  St }\OtherTok{\textless{}{-}}\NormalTok{ S0}
  
  \ControlFlowTok{for}\NormalTok{ (j }\ControlFlowTok{in} \DecValTok{1}\SpecialCharTok{:}\NormalTok{N) \{}
\NormalTok{    epsilon }\OtherTok{\textless{}{-}} \FunctionTok{rnorm}\NormalTok{(}\DecValTok{1}\NormalTok{)}
\NormalTok{    Stn }\OtherTok{\textless{}{-}}\NormalTok{ St }\SpecialCharTok{*} \FunctionTok{exp}\NormalTok{(nudt }\SpecialCharTok{+}\NormalTok{ volsdt }\SpecialCharTok{*}\NormalTok{ epsilon)}
\NormalTok{    St }\OtherTok{\textless{}{-}}\NormalTok{ Stn}
    
    \ControlFlowTok{if}\NormalTok{ (St }\SpecialCharTok{\textgreater{}=}\NormalTok{ H) \{}
\NormalTok{      BARRIER }\OtherTok{\textless{}{-}} \ConstantTok{TRUE}
      \ControlFlowTok{break}
\NormalTok{    \}}
\NormalTok{  \}}
  
  \ControlFlowTok{if}\NormalTok{ (BARRIER) \{}
\NormalTok{    CT }\OtherTok{\textless{}{-}} \DecValTok{0}
\NormalTok{  \} }\ControlFlowTok{else}\NormalTok{ \{}
\NormalTok{    CT }\OtherTok{\textless{}{-}} \FunctionTok{max}\NormalTok{(}\DecValTok{0}\NormalTok{, K }\SpecialCharTok{{-}}\NormalTok{ St)}
\NormalTok{  \}}
  
\NormalTok{  sum\_CT }\OtherTok{\textless{}{-}}\NormalTok{ sum\_CT }\SpecialCharTok{+}\NormalTok{ CT}
\NormalTok{  sum\_CT2 }\OtherTok{\textless{}{-}}\NormalTok{ sum\_CT2 }\SpecialCharTok{+}\NormalTok{ CT}\SpecialCharTok{\^{}}\DecValTok{2}
\NormalTok{\}}

\CommentTok{\# Compute Expectation and SE}
\NormalTok{C0 }\OtherTok{\textless{}{-}} \FunctionTok{exp}\NormalTok{(}\SpecialCharTok{{-}}\NormalTok{r }\SpecialCharTok{*}\NormalTok{ T) }\SpecialCharTok{*}\NormalTok{ sum\_CT}\SpecialCharTok{/}\NormalTok{M}
\NormalTok{sigma }\OtherTok{\textless{}{-}} \FunctionTok{sqrt}\NormalTok{((sum\_CT2 }\SpecialCharTok{{-}}\NormalTok{ sum\_CT}\SpecialCharTok{\^{}}\DecValTok{2}\SpecialCharTok{/}\NormalTok{M) }\SpecialCharTok{*} \FunctionTok{exp}\NormalTok{(}\SpecialCharTok{{-}}\DecValTok{2} \SpecialCharTok{*}\NormalTok{ r }\SpecialCharTok{*}\NormalTok{ T) }\SpecialCharTok{/}\NormalTok{ (M }\SpecialCharTok{{-}} \DecValTok{1}\NormalTok{))}
\NormalTok{SE }\OtherTok{\textless{}{-}}\NormalTok{ sigma }\SpecialCharTok{/} \FunctionTok{sqrt}\NormalTok{(M)}

\FunctionTok{cat}\NormalTok{(}\StringTok{"Call value is $"}\NormalTok{, }\FunctionTok{round}\NormalTok{(C0, }\DecValTok{2}\NormalTok{), }\StringTok{" with SE +/{-} "}\NormalTok{, }\FunctionTok{round}\NormalTok{(SE, }\DecValTok{3}\NormalTok{), }\StringTok{"}\SpecialCharTok{\textbackslash{}n}\StringTok{"}\NormalTok{)}
\end{Highlighting}
\end{Shaded}

\begin{verbatim}
## Call value is $ 7.06  with SE +/-  0.32
\end{verbatim}

\hypertarget{vectorisation}{%
\subsection{Vectorisation}\label{vectorisation}}

\begin{Shaded}
\begin{Highlighting}[]
\CommentTok{\#{-}{-}{-}{-}{-}{-}{-}{-}{-}{-}{-}{-}{-}{-}{-}{-}{-}{-}{-}{-}{-}{-}{-}{-}{-}{-}{-}{-}{-}\#}
\CommentTok{\# Vectorized Implementation   \#}
\CommentTok{\#{-}{-}{-}{-}{-}{-}{-}{-}{-}{-}{-}{-}{-}{-}{-}{-}{-}{-}{-}{-}{-}{-}{-}{-}{-}{-}{-}{-}{-}\#}

\CommentTok{\# Set seed for reproducibility}
\FunctionTok{set.seed}\NormalTok{(}\DecValTok{689}\NormalTok{)}

\CommentTok{\# Precompute constants}
\NormalTok{dt }\OtherTok{\textless{}{-}}\NormalTok{ T}\SpecialCharTok{/}\NormalTok{N}
\NormalTok{nudt }\OtherTok{\textless{}{-}}\NormalTok{ (r }\SpecialCharTok{{-}} \FloatTok{0.5} \SpecialCharTok{*}\NormalTok{ vol}\SpecialCharTok{\^{}}\DecValTok{2}\NormalTok{) }\SpecialCharTok{*}\NormalTok{ dt}
\NormalTok{volsdt }\OtherTok{\textless{}{-}}\NormalTok{ vol }\SpecialCharTok{*} \FunctionTok{sqrt}\NormalTok{(dt)}
\NormalTok{erdt }\OtherTok{\textless{}{-}} \FunctionTok{exp}\NormalTok{(r }\SpecialCharTok{*}\NormalTok{ dt)}

\CommentTok{\# Monte Carlo Method}
\NormalTok{Z }\OtherTok{\textless{}{-}} \FunctionTok{matrix}\NormalTok{(}\FunctionTok{rnorm}\NormalTok{(N }\SpecialCharTok{*}\NormalTok{ M), }\AttributeTok{nrow =}\NormalTok{ N, }\AttributeTok{ncol =}\NormalTok{ M)}
\NormalTok{delta\_St }\OtherTok{\textless{}{-}}\NormalTok{ nudt }\SpecialCharTok{+}\NormalTok{ volsdt }\SpecialCharTok{*}\NormalTok{ Z}
\NormalTok{ST }\OtherTok{\textless{}{-}} \FunctionTok{matrix}\NormalTok{(S0, }\AttributeTok{nrow =}\NormalTok{ N }\SpecialCharTok{+} \DecValTok{1}\NormalTok{, }\AttributeTok{ncol =}\NormalTok{ M)}
\ControlFlowTok{for}\NormalTok{ (i }\ControlFlowTok{in} \DecValTok{2}\SpecialCharTok{:}\NormalTok{(N }\SpecialCharTok{+} \DecValTok{1}\NormalTok{)) \{}
\NormalTok{  ST[i,] }\OtherTok{\textless{}{-}}\NormalTok{ ST[i }\SpecialCharTok{{-}} \DecValTok{1}\NormalTok{,] }\SpecialCharTok{*} \FunctionTok{exp}\NormalTok{(delta\_St[i }\SpecialCharTok{{-}} \DecValTok{1}\NormalTok{,])}
\NormalTok{\}}

\CommentTok{\# Apply Barrier Condition to ST matrix}
\NormalTok{mask }\OtherTok{\textless{}{-}} \FunctionTok{apply}\NormalTok{(ST, }\DecValTok{2}\NormalTok{, }\ControlFlowTok{function}\NormalTok{(x) }\FunctionTok{any}\NormalTok{(x }\SpecialCharTok{\textgreater{}=}\NormalTok{ H))}
\NormalTok{ST[, mask] }\OtherTok{\textless{}{-}} \DecValTok{0}

\NormalTok{CT }\OtherTok{\textless{}{-}} \FunctionTok{pmax}\NormalTok{(}\DecValTok{0}\NormalTok{, K }\SpecialCharTok{{-}}\NormalTok{ ST[N }\SpecialCharTok{+} \DecValTok{1}\NormalTok{, ST[N }\SpecialCharTok{+} \DecValTok{1}\NormalTok{,] }\SpecialCharTok{!=} \DecValTok{0}\NormalTok{])}
\NormalTok{C0 }\OtherTok{\textless{}{-}} \FunctionTok{exp}\NormalTok{(}\SpecialCharTok{{-}}\NormalTok{r }\SpecialCharTok{*}\NormalTok{ T) }\SpecialCharTok{*} \FunctionTok{sum}\NormalTok{(CT) }\SpecialCharTok{/}\NormalTok{ M}

\NormalTok{sigma }\OtherTok{\textless{}{-}} \FunctionTok{sqrt}\NormalTok{(}\FunctionTok{sum}\NormalTok{((}\FunctionTok{exp}\NormalTok{(}\SpecialCharTok{{-}}\NormalTok{r }\SpecialCharTok{*}\NormalTok{ T) }\SpecialCharTok{*}\NormalTok{ CT }\SpecialCharTok{{-}}\NormalTok{ C0)}\SpecialCharTok{\^{}}\DecValTok{2}\NormalTok{) }\SpecialCharTok{/}\NormalTok{ (M }\SpecialCharTok{{-}} \DecValTok{1}\NormalTok{))}
\NormalTok{SE }\OtherTok{\textless{}{-}}\NormalTok{ sigma }\SpecialCharTok{/} \FunctionTok{sqrt}\NormalTok{(M)}

\FunctionTok{cat}\NormalTok{(}\StringTok{"Call value is $"}\NormalTok{, }\FunctionTok{round}\NormalTok{(C0, }\DecValTok{2}\NormalTok{), }\StringTok{" with SE +/{-} "}\NormalTok{, }\FunctionTok{round}\NormalTok{(SE, }\DecValTok{3}\NormalTok{), }\StringTok{"}\SpecialCharTok{\textbackslash{}n}\StringTok{"}\NormalTok{)}
\end{Highlighting}
\end{Shaded}

\begin{verbatim}
## Call value is $ 7.53  with SE +/-  0.309
\end{verbatim}

\end{document}
